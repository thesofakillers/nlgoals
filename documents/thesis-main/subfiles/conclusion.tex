\documentclass[../main.tex]{subfiles}
\begin{document}
\chapter{Conclusion}\label{conc:chap}

In this work, we presented a first attempt at addressing the issue of goal misgeneralization by
focusing on the improvement of task specification through the use of natural language. We devoted
our attention to sequential decision making, and provided a preliminary formalization of a task
specification framework. Here, we identified requester and executor parties, as well as the notion
of a latent specification representation, corresponding to some high-level abstraction of the
desired trajectory. Concurrently, we provide a new definition of the goal misgeneralization
phenomenon, and are the first to explicitly frame it in the context of multi-task learning and to
link it to Occam's razor. We developed our own implementation of a possible solution, first
demonstrating its applicability on a challenging benchmark, and later tackling a simpler environment
platform for toy scenarios of goal misgeneralization. Here, we showed that, under our
implementation, natural language appears to decrease the extent of goal misgeneralization, but
nevertheless fails to completely eliminate the problem. We performed some diagnostic experiments to
understand possible failure modes, and provide some discussions around current limitations and
potential future work.

\ifSubfilesClassLoaded{%
	\bibliographystyle{\subfix{bibstyle}}
	\bibliography{\subfix{references-bibtex}}%
}{}
\end{document}
