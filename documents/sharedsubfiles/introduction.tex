% comment out one of the next two lines accordingly
% \documentclass[../thesis-main/main.tex]{subfiles}
\documentclass[../thesis-proposal/main.tex]{subfiles}
\begin{document}

Of the three Machine Learning (ML) paradigms, Reinforcement Learning (RL) is said to be the closest
to the kind of learning humans do \citep{sutton_reinforcement_2018}, with debates around whether RL
alone is sufficient for solving intelligence \citep{silver_reward_2021, vamplew_scalar_2022}. While
already well-established as a field for decades \citep{minsky_theory_1954, bellman_dynamic_1957,
klopf_brain_1972, bryson_optimal_1996}, the recent widespread adoption of deep neural networks (NNs)
in (un)supervised learning in vision \citep{krizhevsky_imagenet_2012, he_deep_2016,
mildenhall_nerf_2021} and language \citep{mikolov_distributed_2013, vaswani_attention_2017,
devlin_bert_2019, brown_language_2020} has led to similar deep learning (DL) approaches to RL,
achieving impressive results on a wide gamut of problems \citep{mnih_playing_2013,
silver_general_2018, openai_dota_2019, fawzi_discovering_2022}. 

While these academic advances are certainly impressive, RL's inherently online nature has hindered
its integration in contexts where interacting with the environment is either expensive or dangerous,
such as consumer robotics \citep{singh_reinforcement_2022}, healthcare
\citep{liu_reinforcement_2020} and autonomous vehicles \citep{kiran_deep_2022}. In synchrony with
ML's shift to more data-driven approaches, much attention is now being dedicated to
\textit{offline}\footnote{Also known as \textit{batch} RL.} RL \citep{levine_offline_2020,
prudencio_survey_2022}. Rather than interacting with the environment in real-time, offline RL aims
to learn optimal control policies using a fixed dataset of previously collected interactions. Making
use of large, diverse datasets can lead to improved generalization, and still affords the option to
fine-tune the resulting policies online at a later stage, drawing parallels with the ``pre-train
first, then fine-tune'' trend in present-day ML.

Unfortunately, offline RL presents a series of non-trivial issues. In particular, offline RL can be
vulnerable to \textit{distributional shift}, where the data distribution changes between training
and deployment \citep{quinonero-candela_dataset_2008}. This issue of out-of-domain (OOD)
generalization \citep{shen_towards_2021} is pervasive to the current paradigm of ML research
\citep{arjovsky_out_2020}, and is generally exacerbated by the use of high-dimensional function
approximators such as deep neural networks. One of the surprising effects of OOD misgeneralization
is \textit{causal confusion}, where the learner wrongly identifies the causal model of the data due
to spurious correlations \citep{de_haan_causal_2019}. The more classic  failure mode of causal
confusion is \textit{capability misgeneralization}: the model fails to take correct/useful actions
in deployment \citep{gupta_can_2022, tien_study_2022}. Along with \citet{shah_goal_2022},
\citet{langosco_goal_2022} identify \textit{goal misgeneralization} as an additional, potentially
more dangerous failure mode, where the model capably pursues the wrong goal in deployment.
Unsurprisingly, both works suggest more diverse training data as a potential mitigation.

In this vein, we set out to tackle this issue by proposing the use of natural language (NL) as an
additional data modality to be integrated into our offline RL algorithms. Inspired by their
applications beyond a pure NLP context \citep{dosovitskiy_image_2022, ramesh_hierarchical_2022,
rombach_high-resolution_2022}, we can develop a more natural and expressive interface between human
and machine to specify goals. This is after-all how humans communicate desired outcomes to each
other. In short, we aim to address the following questions:

\begin{enumerate}
  \item How can we use natural language to address goal misgeneralization in offline RL? 
  \item Can natural language exacerbate goal misgeneralization?
  \item If so, under which circumstances does natural language alleviate vs.~exacerbate the issue?
\end{enumerate}

We frame these questions in the larger context of goal specification in RL
\citep{white_unifying_2017, liu_goal-conditioned_2022, bansal_specification-guided_2022}:
\textit{When is a goal specification good? What is a feasible, sensible way to specify goals? Can
goal specification methods be used in a complementary manner?} These are some of the more general
questions we wish to address at least partly in this work.

\ifSubfilesClassLoaded{%
  \bibliographystyle{../bibstyle}
  \bibliography{../references-bibtex}%
}{}
\end{document}
