% comment out one of the next two lines accordingly
% \documentclass[../thesis-main/main.tex]{subfiles}
\documentclass[../thesis-proposal/main.tex]{subfiles}
\begin{document}
\section{Ingredients}

To answer any of the research questions outlined in Chapter \ref{chap:intro}, we will require the
general ingredients presented in the following sub-sections:

\subsection{Defining Goal Misgeneralization}

Due to its simplicity and generality, we will rely on \citet{shah_goal_2022}'s definition of goal
misgeneralization. The authors first define a \textit{misgeneralization} framework: consider the
scenario in which we are aiming to learn some function $f^* : X \rightarrow Y$ that maps inputs $x
\in X$ to outputs $y \in Y$. Next, consider the set of parametrized functions $\mathcal{F}_\Theta$
implemented by some arbitrary ML model. We can select a function $f_\theta$ from this set based on
some scoring function $s(f_\theta, \mathcal{D})$ which evaluates how well $f_\theta$ performs on the
dataset $\mathcal{D}$. Misgeneralization can then occur when there are two parametrizations
$\theta_1$, $\theta_2$  such that $f_{\theta_1}$ $f_{\theta_2}$ perform well on one dataset
$\mathcal{D}_{train}$ but whose performances differ on another dataset $\mathcal{D}_{test}$.

Having defined misgeneralization
% todo define goal misgeneralization

\subsection{NL-enabled Offline RL Algorithm(s)}
% - LMPriors
% - no well established work, so would like to try several

\subsection{NL-enabled Offline RL Environments}
% - environments
%   - scale to more complex

\subsection{Measuring Goal Misgeneralization}
% - measuring goal misgeneralization

\section{General Method}

To answer question 1, we expect to simply benchmark our algorithm(s) on our environments with and
without natural language aids. If goal misgeneralization is measured to be lower when using language
than without, we have evidence that natural language can help in this regard, which we hypothesize
to be the case.

To answer question 2, we aim to develop seemingly-harmless but ultimately ambiguous natural language
instructions that may be misinterpreted by the algorithm, similarly to the unreliable wish-granting
genie from the three-wishes parable \citep{perrault_les_1865, galland_les_1717}. We hypothesize that
such language can actually exacerbate goal misgeneralization, and intend to search for evidence of
this by benchmarking our algorithm(s) on our environment(s) across a range of ambiguity.

We have yet to develop concrete methodology for addressing question 3, and we imagine that this will
be developed as work progresses on the first two questions and the research is better crystallized 
in our minds.


\ifSubfilesClassLoaded{%
  \bibliographystyle{../bibstyle}
  \bibliography{../references-bibtex}%
}{}
\end{document}
