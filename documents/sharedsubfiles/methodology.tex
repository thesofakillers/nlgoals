% comment out one of the next two lines accordingly
% \documentclass[../thesis-main/main.tex]{subfiles}
\documentclass[../thesis-proposal/main.tex]{subfiles}
\usepackage{parskip}
\begin{document}
\section{Ingredients}

To answer any of the research questions outlined in Chapter \ref{chap:intro}, we will require the
general ingredients presented in the following sub-sections:

\subsection{Defining Goal Misgeneralization}

Due to its simplicity and generality, we will rely on \citet{shah_goal_2022}'s definition of goal
misgeneralization. The authors first define a \textit{misgeneralization} framework: consider the
scenario in which we are aiming to learn some function $f^* : X \rightarrow Y$ that maps inputs $x
	\in X$ to outputs $y \in Y$. Next, consider the set of parametrized functions $\mathcal{F}_\Theta$
implemented by some arbitrary ML model. We can select a function $f_\theta$ from this set based on
some scoring function $s(f_\theta, \mathcal{D})$ which evaluates how well $f_\theta$ performs on the
dataset $\mathcal{D}$. Misgeneralization can then occur when there are two parametrizations
$\theta_1$, $\theta_2$  such that $f_{\theta_1}$ $f_{\theta_2}$ perform well on one dataset
$\mathcal{D}_{train}$ but whose performances differ on another dataset $\mathcal{D}_{test}$.

Having defined misgeneralization, the authors rely on two additional definitions two define goal
misgeneralization. Firstly, the authors refer to a model as \textit{capable} of some task $Z$ in
setting $W$ if it can be quickly tuned to perfrom $Z$ in $W$. Secondly, a model's behaviour is
\textit{consistent} with the goal of performing task $Z$ in setting $W$ if its behaviour in setting
$W$ shows good performance on task $Z$. Goal misgeneralization then occurs if in setting $W_{test}$,
the model is capable of the intended goal of performing task $Z_{i}$ but its behaviour is
instead consistent with some other goal of performing task $Z_{m}$.

\subsection{NL-enabled Offline RL Algorithm(s)}

In principle, we envision our solution architecture consisting in the combination of a grounded
language encoder and an offline RL algorithm such as BCQ or CQL, with the environment state
perceived visually through pixels. The language encoder is ``grounded'' in the sense that for
a given piece of text relating to some state of the environment, the resulting text embedding is in
a similar semantic space as the visual embedding of that state. This is most similar to the work by
\citet{fan_minedojo_2022}, which fine-tuned CLIP \citep{radford_learning_2021} on their specific
environmental domain. Work is currently ongoing at the Fund for Alignment Research
(FAR)\footnote{\href{https://far.ai/}{https://far.ai/}} on extending this approach such that no
fine-tuning is necessary, so that the same CLIP instance can be used on a variety of environments
through a few prompting ``shots''\footnote{Juan Rocamonde, the external supervisor for this thesis,
is currently working on this line of research at FAR.}. We believe that in the context of offline
RL, using NL priors to avoid goal misgeneralization is an intuitive solution, so we are considering
adapting \citet{choi_lmpriors_2022}'s LMPriors algorithm to the offline RL setting. However, because
generally there is no well established solution for NL-informed, we may ultimately try several
different approaches.

\subsection{NL-enabled Offline RL Environments for Goal Generalization}

We view goal misgeneralization as one of the consequences of causal confusion, where the learned
causal models are erroneous due to spurious correlations in the training data. As such, each of our
environments and tasks will be purposely constructed such that training presents spurious
correlations between our scoring function, one or more irrelevant variable and the actually relevant
predictive variables. Each of these will most likely be unique to the environment.

Similarly to \citet{langosco_goal_2022}, we extend the Procgen~\citep{cobbe_leveraging_2020}
benchmark environment suite for our goal misgeneralization setting. Unlike
\citet{langosco_goal_2022} however, our extensions also include NL instructions, feedback and/or
labels, depending on which algorithm we are considering. These NL features are algorithmically
generated through scripted templates and potentially language models, to avoid reliance on human
labellers which go beyond the scope of this work.

For more complex environments, we may eventually move from Procgen to MineDojo
\citep{fan_minedojo_2022}, which already presents NL annotations but may require additional
scripting to build datasets for offline RL. In general, we expect to have a better idea of what
environment desiderata to consider as work progresses.

\subsection{Measuring Goal Misgeneralization}

Measuring goal misgeneralization can be tricky and is generally environment-specific. Like previous
works \citep{langosco_goal_2022, shah_benefits_2021, de_haan_causal_2019}, we compare the
performance (reward) with and without spurious correlations with confounding variables to measure
the extent of goal misgeneralization. A larger drop in performance will indicate more severe goal
misgeneralization. To distinguish this failure mode from capability misgeneralization, we will also
track the confounding variable when it is and isn't spuriously correlated to the intended goal.
Depending on the environment, if we note similar extents of interaction with the confounding
variable with and without the spurious correlation, we may be able to conclude that the agent is
still acting capably, but simply pursuing the wrong goal. As mentioned, we note that these
quantifications are mostly environment-specific. We hope to develop a more general method of
measuring goal misgeneralization, but like other aspects of our method, we cannot provide more
detail to this end at the moment.

\section{General Method}

To answer question 1, we expect to simply benchmark our algorithm(s) on our environments with and
without NL aids. If goal misgeneralization is measured to be lower when using language
than without, we have evidence that NL can help in this regard, which we hypothesize
to be the case.

To answer question 2, we aim to develop seemingly-harmless but ultimately ambiguous NL
instructions that may be misinterpreted by the algorithm, similarly to the unreliable wish-granting
genie from the three-wishes parable \citep{perrault_les_1865, galland_les_1717}. We hypothesize that
such language can actually exacerbate goal misgeneralization, and intend to search for evidence of
this by benchmarking our algorithm(s) on our environment(s) across a range of ambiguity.

We have yet to develop concrete methodology for addressing question 3, and we imagine that this will
be developed as work progresses on the first two questions and the research is better crystallized
in our minds.


\ifSubfilesClassLoaded{%
	\bibliographystyle{../bibstyle}
	\bibliography{../references-bibtex}%
}{}
\end{document}
